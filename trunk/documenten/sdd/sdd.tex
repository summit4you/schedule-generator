\documentclass{article}
\usepackage[dutch]{babel}
\usepackage[official]{eurosym}
\usepackage{graphicx}
\usepackage{epic,eepic}
\usepackage{mathtools}
\usepackage{caption}
\usepackage{float}
\usepackage{hyperref}
\usepackage{textcomp}
\setcounter{secnumdepth}{5}
\graphicspath{{fig/}}

\makeatletter
\renewcommand\paragraph{%
   \@startsection{paragraph}{4}{0mm}%
      {-\baselineskip}%
      {.5\baselineskip}%
      {\normalfont\normalsize\bfseries}}
\makeatother

\title{Software Design Descriptions voor Schedule-Generator}
\author{Matthias Caenepeel \and Adam Cooman \and Alexander De Cock \and Zjef Van de Poel}
\date{28 maart 2011 Versie 2.0} 

\begin{document}

\maketitle

\newpage

% \newpage
% Signature page

% \newpage

\section*{Aanpassingsgeschiedenis}
\begin{itemize}
\item[.] 23/2/2011 versie 0.1: Aanmaak document 
\item[.] 27/2/2011 versie 0.2: Toevoeging delen over interfaces 
\item[.] 28/2/2011 versie 0.3: Toevoeging hoofdstuk over algoritme en Logical
\item[.] 28/2/2011 versie 1.0: Verbeteringen doorgevoerd 
\item[.] 17/3/2011 versie 1.1: Opmerkingen opdrachtgever in acht genomen en nodige aanpassingen gedaan. Gebruik van de GET en POST methode in de Site - Servlet interface grondiger uitgelegd 
\item[.] 28/3/2011 versie 2.0: Volledige revisie van het document en aanpassingen/toevoegingen doorgevoerd waar nodig.
\end{itemize}

\section*{Nog te doen}
\begin{itemize}
\item[.] Sectie 1.5 is nog wat karig, hoewel de inhoud die er is alvast interessant is. Wat ontbreekt is nog het eigenlijke design. \\[-3mm]
\item[.] Component Diagram toevoegen aan design viewpoint 1 \\[-3mm]
\end{itemize}
\item[.] UML diagrammas van Alexander toevoegen waar nodig
%\section*{Preface}

\newpage
\tableofcontents

%\newpage
%\section{Introduction}
%TODO
%\subsection{Purpose}
%\subsection{Scope}
%\subsection{Context}
%\subsection{Summary}

%\newpage
%\section{References}

%\newpage
%\part{Glossary}

\newpage
\section{Body}
%\subsection{Identified stakeholders and design concerns}

\subsection{Design viewpoint 1: Compositie}

\subsubsection{Design concerns}
Het doel van dit ontwerpstandpunt bestaat er in alle componenten van het systeem te identificeren, hun attributen te kenmerken en hun onderlinge verbanden te beschrijven. Deze beschrijving zal gebeuren aan de hand van een \textit{component diagram} en het \textit{deployment diagram}. Beide  zijn een onderdeel van de UML 2.0 modelleertaal.  

\subsubsection{Deployment Diagram}

Een deployement diagram geeft weer op welke manier de hard- en software wordt geconfigureerd tijdens de normale werking van het systeem. Hieronder volgt een beschrijving van de belangrijkste componenten in het diagram. 

\begin{figure}[H]%
\includegraphics[width=\columnwidth]{DeploymentDiagram.png}%
\caption{Deployment Diagram}%
\end{figure}

\begin{description}

\item[Gebruiker] Een gebruiker die beschikt over een account kan zich aanmelden op de website via zijn gebruikersnaam en wachtwoord. Vervolgens zal hij beschikken over de functionaliteiten, eigen aan zijn gebruikertype. Als de gebruiker geen account heeft kan hij de site betreden als gast. Er wordt dan geen gebruikersgebonden informatie bijgehouden. Meer informatie over de verschillende gebruikertypes en functionaliteiten is voorzien in het SRS. \\
 
\item[Persoonlijke Computer] Toestel dat de hard- en software aan de gebruikerszijde bevat. Er worden geen onderstellingen gemaakt over de eigenschappen van dit toestel met uitzondering dat het instaat is een webbrowser te draaien met eigenschappen gelijkaardig aan die van Internet Explorer, Google Chrome of FireFox.  \\
 
\item[Webbrowser] Er worden geen specifieke onderstelling gemaakt met uitzondering van de reeds vermelde. \\
 
\item[Systeemserver] Tijdens de ontwikkeling van het project wordt Wilma als systeemserver gebruikt. De specificaties van de server worden volledig bepaald door de opdrachtgever en kunnen terug gevonden worden op \\ http://wilma.vub.ac.be/. \\
 
\item[Tomcat] Dit is een webcontainer ontwikkeld door Apache die onder andere toelaat servlets te draaien op een Linux server. Deze servlets zullen de vragen van de gebruiker opvangen en op dynamische wijze webinhoud generen als antwoord. Deze webinhoud zal hoofdzakelijk worden beschreven via XHTML en CVS. Voor meer informatie over Tomcat kan men terecht op \\ http://tomcat.apache.org/. \\
 
\item[Database] De database  maakt onderdeel van de hardware van de systeemserver en bevat zowel de account informatie als de informatie die relevant is voor het opstellen van de lessenroosters. Om gegevens uit de database toegankelijk te maken voor elementen uit de Java runtime environment (JRE) zal gebruikt worden gemaakt van JDBC. Dit is een Java Application Programming Interface ontwikkeld door Sun. Een overzicht wordt gegeven op \\ www.oracle.com/technetwork/java/javase/tech/index-jsp-136101.html \\

\end{description}


%\subsubsection{Component Diagram}

%Zal later worden toegevoegd


%-----------------------------------------------------------------------------------------------------------------------------------------------------

\subsection{Design viewpoint 2: Logical}
\subsubsection{Design concerns}
Om de lessenrooster op te stellen en uit te lezen zijn er verschillende klasses nodig die alle verschillende participerende objecten voorstellen.

\subsubsection{Elementen}
\paragraph{Entiteiten}

\begin{itemize}
	\item[-] Lessen structuur \\[-5mm]
	\begin{itemize}
	  \item[-] \textit{Course}: Les die gevolgd kan worden \\ [-5mm]
	  \item[-] \textit{Subcourse}: Onderdeel van een les; bv hoorcollege, labo of oefeningenles \\[-5mm]
	  \item[-] \textit{Program}: Verzameling van lessen die in een pakket zitten. Bijvoorbeeld 1e bachelor ingenieurswetenschappen \\[-5mm]
	\end{itemize}
	\item[-] Gebouwen strucuur \\[-5mm]
	\begin{itemize}
	  \item[-] \textit{Building}: Gebouw \\[-5mm]
    \item[-] \textit{Room}: Lokaal in een gebouw waar les gegeven kan worden \\[-5mm]
    \item[-] \textit{Hardware}: Materiaal beschikbaar in een lokaal \\[-5mm]
	\end{itemize}
	\item[-] Personen structuur \\[-5mm]
	\begin{itemize}
	  \item[-] \textit{Student}: Persoon die lessen kan volgen \\[-5mm]
    \item[-] \textit{Educator}: Persoon die lessen kan geven; zowel professoren, docenten en assistenten \\[-5mm]
	\end{itemize}
	\item[-] Andere \\[-5mm]
	\begin{itemize}
	  \item[-] \textit{Faculty}: Faculteit van de universiteit \\[-5mm]
	\end{itemize}
\end{itemize}

\paragraph{Relaties}

\begin{itemize}
\item[-] Lessen structuur \\[-5mm]
	\begin{itemize}
	\item[] Elke \textit{Course} bestaat uit een of meerdere \textit{SubCourses} die in opgeteld het gehele vak weergeven. \\[-5mm]
	\item[] \textit{Courses} zelf worden gegroepeerd in een \textit{Program}. \\
	\end{itemize}
\item[-]Gebouwen structuur \\[-5mm]
\begin{itemize}
	\item[] Een gebouw beschikt over een lijst met \textit{Rooms}, die op zich een lijst met beschikbare \textit{Hardware} bijhoudt. \\[-5mm]
	\item[] Een gebouw kan ingedeeld worden onder een faculteit. \\[-5mm]
	\end{itemize}
\item[-] Personen \\[-5mm]
	\begin{itemize}
	\item[] Elke \textit{Student} is ingeschreven in een \textit{Program} en een lijst met afzonderlijke \textit{Courses} die ze volgen.  \\[-5mm]
	\item[] Een \textit{Educator} beschikt over een lijst \textit{Courses} en \textit{SubCourses} die ze geven. \\[-5mm]
\end{itemize}		
\end{itemize}

\paragraph{UML Diagrammma}

\begin{figure}%
\includegraphics[width=\columnwidth]{UML.png}%
\caption{UML Diagramma}%
\label{UML}
\end{figure}

De klasse waarbij de pijl vertrekt erft van de klasse waar de pijl aankomt. De klasse waarbij er src staat zit in een lijst in de klasse waarbij er dst staat. Zo zit er bijvoorbeeld een lijst van educators in de klasse faculty. \\


\newpage


%------------------------------------------------------------------------------------------------------------------------------------------------------------------


\subsection{Design viewpoint 3: Interfaces}
% HTTP
% Adam schrijft dit het is toegelaten dat je zelf de punten een beke invult

% HTML bestand met javascript om tabs te maken
% HTML bestand met CSS voor layout
% Javascript met CSS intern voor tabber
% HTML met server via HTTP (GET en POST)
% Servlet via interface van Zjef en MySQL met Database

\subsubsection{Design concerns, algemeen}

\begin{figure}[H]
\unitlength=0.3\textwidth
\begin{center}
\begin{picture}(3,1)
%serverlijnen
\put(0.2,0.2){Server}
\put(0.0,0.0){\line(1,0){1.6}}
\put(0.0,0.0){\line(0,1){1.0}}
\put(1.6,0.0){\line(0,1){1.0}}
\put(0.0,1.0){\line(1,0){1.6}}
%browser
\put(2.4,0.2){Browser}
\put(2.2,0.0){\line(1,0){1.6}}
\put(2.2,0.0){\line(0,1){1.0}}
\put(3.8,0.0){\line(0,1){1.0}}
\put(2.2,1.0){\line(1,0){1.6}}
%XHTML
\put(2.32,0.7){Webpagina}
\put(2.4,0.5){XHTML}
\put(2.3,0.9){\line(1,0){0.5}}
\put(2.8,0.4){\line(0,1){0.5}}
\put(2.3,0.4){\line(0,1){0.5}}
\put(2.3,0.4){\line(1,0){0.5}}
%CSS
\put(3.0,0.7){Stylesheet; CSS}
\put(2.95,0.8){\line(1,0){0.8}}
\put(2.95,0.65){\line(0,1){0.15}}
\put(2.95,0.65){\line(1,0){0.8}}
\put(3.75,0.65){\line(0,1){0.15}}
% CSS naar XHTML
\put(2.82,0.725){\line(1,0){0.11}} 
%Javascript
\put(3.0,0.5){Script; Javascript}
\put(2.95,0.6){\line(1,0){0.8}}
\put(2.95,0.45){\line(0,1){0.15}}
\put(2.95,0.45){\line(1,0){0.8}}
\put(3.75,0.45){\line(0,1){0.15}}
% Javascript naar XHTML
\put(2.82,0.525){\line(1,0){0.11}} 
%database
\put(0.15,0.6){Database}
\put(0.1,0.9){\line(1,0){0.5}}
\put(0.6,0.4){\line(0,1){0.5}}
\put(0.1,0.4){\line(0,1){0.5}}
\put(0.1,0.4){\line(1,0){0.5}}
%servlet
\put(1.1,0.8){Servlet}
\put(1.1,0.6){JAVA}
\put(1.0,0.95){\line(1,0){0.5}}
\put(1.5,0.55){\line(0,1){0.4}}
\put(1.0,0.55){\line(0,1){0.4}}
\put(1.0,0.55){\line(1,0){0.5}}
% Files
\put(1.1,0.25){Files}
\put(1.1,0.1){XML}
\put(1.0,0.4){\line(1,0){0.5}}
\put(1.5,0.05){\line(0,1){0.35}}
\put(1.0,0.05){\line(0,1){0.35}}
\put(1.0,0.05){\line(1,0){0.5}}
% lijn Files naar Servlets
\put(1.25,0.42){\line(0,1){0.11}} 
%lijn MySQL
\put(0.65,0.7){MySQL}
\put(0.62,0.65){\line(1,0){0.36}}
%lijn HTTP
\put(1.8,0.7){HTTP}
\put(1.52,0.65){\line(1,0){0.76}}
\end{picture}\\[3mm]
\caption{Overzicht van de verschillende delen van het programma}
\end{center}
\end{figure}

In de structuur van ons programma bestaan verschillende elementen met verschillende taken. Deze moeten met elkaar interageren volgens het bovenstaande schema. De lijnen tussen de verschillende blokken noemen we interfaces en zullen in het volgende deel van het Software Design Document besproken worden. Vooraleer daarmee te beginnen een kort overzicht van de taken die elk blok uit het diagramma uitvoert.

\begin{description}
  \item[Webpagina, XHTML] \hfill \\
  In het XHTML bestand wordt de inhoud van de webpagina geplaatst. 
  \item[Stylesheet, CSS] \hfill \\
  In het stylesheet wordt de lay-out van de webpagina beschreven.
  \item[Script, Javascript] \hfill \\
  In het script word beschreven wanneer welk deel van de webpagina weergegeven wordt. We gebruiken tabbladen om de functionaliteiten die een gebruiker krijgt duidelijk weer te geven. Ht beheer van die tabbladen gebeurt met het javascript "tabber" dat ontwikkeld werd door derden;
  \item[Servlet, JAVA] \hfill\\
  De servlets runnen op tomcat en genereren de XHTML code, afhankelijk van de eigenschappen van de gebruiker.
  \item[Database] \hfill\\
  In de database wordt de informatie van gebruikers en de kalender opgeslaan.
  \item[Files] \hfill\\
  De files bevatten wijzigbare parameters van het programma. Ze zijn in een XML bestand opgeslaan.
\end{description}

Het schema is geen volledig correcte weergave van de werkelijkheid omdat het stylesheet en het script zich ook op de server bevinden en door de browser opgehaald worden van de server via een HTTP protocol. Om volledig correct te zijn zouden die twee onderdelen van het schema zich dus ook op de server moeten bevinden. Om alles overzichtelijk te houden heeft de auteur beslist om ze bij de browser te plaatsen, omdat de browser het ophalen van de server voorziet en niet de gebruiker.

\subsubsection{Interface:  XHTML - CSS voor Layout}
\paragraph{Initialiseren} 

Om het .css bestand aan de XHTML pagina te linken moet in de header van de XHTML code het volgende voorzien worden
\begin{verbatim}
<link rel="stylesheet" href="style.css" type="text/css"> 
\end{verbatim}
Hierin zijn de volgende elementen te herkennen:
\begin{itemize}
\item[] \texttt{rel="stylesheet"}: Duidelijk maken dat de link een link naar een stylesheet is. Deze tag verandert niet \\[-5mm]
\item[] \texttt{type="text/css"}: Duidelijk maken dat de stylesheet in css code geschreven is. Deze tag verandert ook niet \\[-5mm]
\item[] \texttt{href="style.css"}: De naam van het css bestand. Als die zich op een andere locatie bevindt dan de XHTML pagina moet het pad naar die map hierin toegevoegd worden \\[-5mm]
\end{itemize}

\paragraph{Gebruiken} 

In de XHTML code moet niet veel toegevoegd worden om de css op te roepen, enkel een id tag op de volgende manier om onderscheid te maken tussen verschillende gedefini\"{e}erde stijlen in de css code. Als voorbeeld wordt het toevoegen van een id aan een rij van een tabel gegeven om aan te tonen hoe dit moet.
\begin{verbatim}
<tr id="MainBottom">
\end{verbatim}
De naam van het id staat tussen de aanhalingstekens.\\

In het CSS bestand kan de code voor de stijl van hetzelfde id geschreven worden door gebruik te maken van hekje. Als voorbeeld de CSS code die de stijl beschrijfd van de tabelrij uit het vorige voorbeeld.
\begin{verbatim}
tr#MainBottom {
	height:300px;
}
\end{verbatim}

\subsubsection{Interface: XHTML - Javascript voor tabbladen}
\paragraph{Initialiseren} 

Het oproepen van het javascript bestand gebeurt in de header van het XHTML bestand dat de inhoud van de site beschrijft met de volgende code.
\begin{verbatim}
<script type="text/javascript" src="tabber-minimized.js"></script>
\end{verbatim}
Hierin zijn de volgende elementen te herkennen:
\begin{itemize}
\item[] \texttt{type="text/javascript"}: Duidelijk maken dat het javascript is. Deze tag verandert niet \\[-5mm]
\item[] \texttt{src="tabber-minimized.js"}: De naam van het javascript bestand. Als die zich op een andere locatie bevindt dan de HTML pagina moet het pad naar die map hierin toegevoegd worden \\[-5mm]
\end{itemize}

\paragraph{Gebruiken}
Om dan een tabblad aan te maken en er inhoud in te plaatsen wordt gebruik gemaakt van div elementen in de XHTML code. dit gebeurd op de volgende manier:
\begin{verbatim}
<div class="tabber">
<div class="tabbertab" title="Tabblad 1">
Inhoud van het eerste tabblad
</div>
<div class="tabbertab" title="Tabblad 2">
Inhoud van het tweede tabblad
</div>
</div>
\end{verbatim}
In de buitenste div staat het volgenden:
\begin{itemize}
\item[]  \texttt{class="tabber"}: dit vertelt aan het javascript dat er tabbladen volgen. Het laat toe om geneste tabbladen te gebruiken \\[-5mm]
\end{itemize}
In de binnenste divs, \'{e}\'{e}n per tabblad:
\begin{itemize}
\item[] \texttt{class="tabbertab"}: dit vertelt aan het javascript dat er tabbladen volgen. Het laat toe om geneste tabbladen te gebruiken \\[-5mm]
\item[] \texttt{title="Tabblad 1"}: In het title tag staat de titel van het tabblad die bovenaan weergegeven wordt \\[-5mm]
\end{itemize}

\subsubsection{Interface: Javascript - CSS voor tabbladen}

Het Javascript tabber dat gebruikt wordt bevat interne links naar het CSS bestand. De exacte werking van de interface is niet gekend omdat tabber een programma is dat geschreven is door derden. De nodige css code werd aan het stijlbestand toegevoegd zodat alles werk. Het is nu mogelijk om de layout van de tabbladen te defini\"{e}ren in het stijlbestand. De nodige onderverdelingen die toegevoegd moeten worden zijn de volgende:

\begin{verbatim}
.tabberlive .tabbertabhide {}
.tabber {}
.tabberlive {}
ul.tabbernav{}
ul.tabbernav li {}
ul.tabbernav li a {}
ul.tabbernav li a:link {}
ul.tabbernav li a:visited {}
ul.tabbernav li a:hover{}
ul.tabbernav li.tabberactive a{}
ul.tabbernav li.tabberactive a:hover{}
.tabberlive .tabbertab {}
\end{verbatim}


\subsubsection{Interface: Browser - Server: HTTP}

\paragraph{Concerns}
Omdat de algemene kalender en andere gegevens in een database op de server zullen opgeslaan zijn en omdat de gebruiker vanop zijn computer thuis via de website die informatie te zien moet krijgen, moet er een communicatie tussen de server en de browser van de gebruiker zijn. Zoals bij de meeste websites werd gekozen om het het HTTP protocol te werken. We laten niet toe dat de gebruikers rechtstreeks met de database communiceren om te vermijden dat ze informatie te zien krijgen die ze niet mogen zien, en om de informatie op een overzichtelijke manier te structureren. Daarom werken we met Servlets die op server runnen en die de besproken taken verrichten. De volgende paragrafen beschrijven dus de communicatie tussen de browser van de gebruiker en de servlets.

\paragraph{Attributes}

Uit het HTTP protocol zullen de GET en POST functies gebruikt worden door onze toepassing. Een HTTP pakket bestaat uit twee delen, de \textit{header} en de \textit{body}.
\begin{description}
\item[GET] Deze methode vraagt een bepaalde pagina aan de server en zet alles in de \textit{header}. Het is de methode die gebruikt wordt bij het klikken op een link van een site. Deze zal dus de meest gebruikte methode zijn om te communiceren met de server. \\
\item[POST] Deze methode stuurt informatie door in de \textit{body} van het paket.\\
\end{description}
De manier waarop de informatie doorgestuurd zal worden hangt af van de plaats. Gevoelige informatie, zoals wachtwoorden zal via een POST verzonden worden, omdat de header gemakkelijk zichtbaar is voor gebruikers. Lange informatie, zoals beschrijvingen van vakken zal ook via de POST methode verzonden worden, omdat de lengte van de \textit{header} beperkt is. \\

Andere informatie, zoals het opvragen van het lessenrooster zal via de GET methode gebeuren. De waarden van de zoekactie worden dan in de link geplaatst door een script dat nog ontwikkeld moet worden en waarvan de interface met de XHTML pagina gelijkenissen zal vertonen met die voor de tabbladen. \\

\paragraph{GET}

We gebruiken HttpServlets die de doGet methode bevatten om een Get aanvraag binnen te krijgen en die te beantwoorden. Beschouw als voorbeeld de volgende servlet. Hij stuurt het antwoord Hallo terug naar de gebruiker als die de juiste URL ingegeven heeft. Welke URL dat is wordt beschreven in een XML bestand dat zich op de server bevindt en een bepaalde URL aan een Servlet koppelt.

\begin{verbatim}
public class test2 extends HttpServlet {
public void doGet(HttpServletRequest request,
                  HttpServletResponse response)
  throws ServletException, IOException {
    response.setContentType("text/html");
    PrintWriter out = response.getWriter();
    out.println("Hallo");
  }
}	
\end{verbatim}

Om variabelen mee te geven naar de servlet met de GET methode wordt gerbuik gemaakt van een vraagteken. Als een variabele \textit{ID} meegegeven wordt aan een URL gebeurt dit op de volgende manier:
\begin{verbatim}
servletnaam?ID=waarde
\end{verbatim}

in de doGet methode kan de waarde van de variabele opgeroepen worden met de getParameter methode:
\begin{verbatim}
String ID = request.getParameter("ID");
\end{verbatim}


\paragraph{POST}
Bij het gebruik van een GET om variabelen aan de servlet door te geven staan de variabelen en hun waarden allemaal in de URL. Voor gevoelige informatie zoals wachtwoorden, of als er zeer veel informatie doorgegeven moet worden (de maximum lengte van een URL in internet explorer is 2083 tekens) is deze methode niet geschikt. Daarvoor zal de POST methode gebruikt worden.\\

Er zal dus enkel gebruik gemaakt worden van de POST methode bij het verzenden van een HTML Form. Deze HTML structuur bevat velden die ingevuld kunnen worden door de gebruiker en een knop waar de gebruiker op moet klikken om de informatie door te sturen. Om een variabele PASS door te sturen met een POST naar de servlet zou de code voor de form er als volgt uitzien:
\begin{verbatim}
<FORM METHOD="POST" ACTION="servlet">
<P> 
<INPUT TYPE="PASSWORD" NAME="PASS"><BR>
<INPUT TYPE="submit" VALUE="Send">
</P>
</FORM>
\end{verbatim}
In de begintag van het form wordt de opstuurmethode gedefinieerd in het METHOD attribuut. deze kan de waarden POST en GET aannemen. In het ACTION attribuut wordt de locatie van de servlet waar de form naartoe gestuurd wordt gespecifieerd.\\
Het inputveld waar de gebruiker de data moet invoeren wordt gemaakt met een INPUT tag. Het attribuut NAME definieert de naam van de variabele en het attribuut TYPE definieert de vorm van het invoerveld\footnote{dit kan veel verschillende vormen aannemen, zie \href{http://www.w3.org/TR/html401/interact/forms.html}{www.w3.org}}.\\
Als laatste essentieel deel van de Form is er de knop waar de gebruiker moet op klikken om de gegevens door te sturen. Dit is opnieuw een INPUT tag, maar nu van het type submit. in het Value attribuut staat de tekst die op de knop moet weergegeven worden.\\

In de servlet kan de doPost methode geimplementeerd worden om de verschillende forms te verwerken. Opnieuw kunnen de variabelen opgevraagd worden met de getParameter methode:
\begin{verbatim}
String ID = request.getParameter("PASS");
\end{verbatim}

Om meer variabelen mee te sturen met een form, is het mogelijk om onzichtbare INPUT tags te defini\"{e}ren in het HTML Form. daarmee kan aan de Servlet bijvoorbeeld meegegeven worden welk Form ingevuld is. 


\subsubsection{Interface: Database interface}

\paragraph{Concerns}
De interface met de MySQL database wordt geleverd door de reeds bestaande java library MySQL Connector/J1 \footnote{http://dev.mysql.com/downloads/connector/j/}.
Deze biedt de mogelijkheid om queries uit te voeren vanuit Java code en de resultaten van deze query uit te lezen.

Om eenvoudig en flexibel te kunnen werken met de database vanuit de object geori�nteerde code is er een extra interface geschreven rond de MySQL Connector.
Deze levert op een eenvoudig te implementeren manier de mogelijkheid om rechtstreeks Java objecten op te slaan en deze ook dusdanig als objecten uit te lezen, met inbegrip van alle onderlinge verbanden tussen de objecten.

\paragraph{design}

\begin{figure}[H]%
\includegraphics[width=\columnwidth]{DATA1.jpg}%
\caption{Overzicht}%
\end{figure}

\begin{figure}[H]%
\includegraphics[width=\columnwidth]{DATA2.jpg}%
\caption{Overzicht van belangrijkste klasses en methodes}%
\end{figure}

De interface biedt communicatie met de database op basis van objecten die aangeboden of opgevraagd worden aan de interface. De interface zal dan zelf de nodige informatie extraheren vanuit het object om het in de database op te slaan, of zal de juiste informatie in het object steken om het uit de database te lezen.
Om deze data extractie correct te laten gebeuren, moet het object aan een bepaald protocol voldoen.
Dit protocol dient zich aan onder de vorm van de java interface klasse 'Databasable'.
Het object dient deze klasse te implementeren.

De getters en setters van de parameters die in en uit de database gehaald moeten worden, moeten worden aangeduid met een Annotation. De interface zal tijdens het vertalen via Reflection in Java zo de juiste methodes kunnen selecteren uit de klasse.
Meer bepaald:
\begin{itemize}
	\item[-] Getters annoteren met @InDatabase
	\item[-] Setters annoteren met @OutDatabase
\end{itemize}


Tijdens het wegschrijven van een object worden alle parameters die teruggegeven worden door de @InDatabase getters opgeslagen in de database.
Tijdens het uitlezen worden alle @OutDatabase setters opgeroepen met de waarde die uitgelezen werd uit de database.

Elk object krijgt een unieke ID (uniek binnen zijn klasse) wanneer hij naar de Database wordt geschreven. Hierdoor moet een parameter van de klasse 'ID' bijgehouden worden. 
De gebruiker dient deze slechts te voorzien  samen met de methodes van de interface Databasable; de invulling van deze ID gebeurt automatisch.

Door dit protocol wordt het schrijven van een object naar de database vereenvoudigd tot het schrijven van:
\begin{verbatim}
myDatabase.write(myObj);
\end{verbatim}

Het uitlezen gebeurt via een 'Search' object, waarmee een bepaald zoek criterium kan opgesteld worden:
\begin{verbatim}
myObj=myDatabase.read(mySearch);
\end{verbatim}

De interface zorgt er ook voor dat de links tussen verschillende objecten ook mee afgehandeld worden. Zie de voorbeelden voor meer detail.
Alle voorwaarden om de Databasable interface correct te implementeren zijn te vinden in de Javadoc commentaar van Databasable.

Voorbeelden van correcte implementatie zijn te vinden in de google-code repository bij:
trunk/Database/Voorbeeld

%\paragraph{Elementen}
%
%\begin{itemize}
%\item[-] Database \\[-5mm]
%\begin{itemize}
%\item[-] \textbf{Database}: Handle naar de database. Levert methodes om tabellen in te lezen, opzoekingen te 		doen, objecten weg te schrijven en dergelijke.
%\item[-] \textbf{Table}: Een enkele tabel uit de database
%\item[-] \textbf{Element}: Een element van een tabel
%\end{itemize}		
%\item[-] Interface \\[-5mm]
%\begin{itemize}
%\item[-] \textbf{Databasable}: Inteface* die duidelijk maakt dat een object van een klasse die 	\textit{Databasable} implementeert, in de database geschreven, of uit de database gelezen 	kan worden. \\[-5mm]
%\item[-] \textbf{InDatabase}: Annotation* die duidelijk maakt dat een bepaalde parameter van een object in de 		database bewaard moet worden. \\[-5mm]
%\item[-] \textbf{OutDatabase}: Annotation* die duidelijk maakt dat een bepaalde parameter van een object  uit de 		database gelezen en naar het object gekopieerd moet worden. \\[-5mm]
%\end{itemize}		
%\end{itemize}		
%		
%*: Java terminologie

\subsubsection{Interface: XML interface}

\paragraph{Concerns}

Er bestaan reeds talloze XML parsers, maar omwille van flexibiliteit en modificeerbaarheid is er een eigen parser geschreven.
Zo is het huidige format niet meer zuiver xml, maar zijn extra features toegevoegd die extra functionaliteiten toevoegen, maar op een manier dat 'echte' xml parsers de door deze parser gegenereerde xml nog steeds kan weergeven.

\paragraph{Design}

Een boomstructuur van java objecten wordt opgesteld vanuit een xml bestand, of omgekeerd.
Uitgedrukt in BNF:
\begin{verbatim}
<XMLDocument> ::= {<XMLElement>}
<XMLElement> ::= <ElementWithValue> | <ElementWithChildren>
<ElementWithChildren> ::= <name><value>
<ElementWithChildren> ::= <name>{<XMLElement>}
<name> ::= {<any_character_no_space>}
<value> ::= {<any_character>}
\end{verbatim}

\begin{figure}[H]%
\includegraphics[width=\columnwidth]{XML1.jpg}%
\caption{Voorbeeld van XMLDocument}%
\end{figure}

\paragraph{Features}
\textbf{Tags}
Specifieke tags kunnen toegevoegd worden aan de elementen:
\begin{verbatim}
<element tag=XXX> ... </element tag=XXX>
\end{verbatim}

\textbf{Code tag}
Door deze tag toe te voegen, wordt alle opmaak binnenin het element genegeerd. Dit kan gebruikt worden wanneer men bijvoorbeeld html code wil opslaan in een xml-element.
Normaal gezien zouden de html tags interfereren met de xml tags, maar de code-tag zorgt ervoor dat alles wat binnen de tags van het element staat gewoon als tekst aanzien wordt
\textbf{Link tag}
Deze tag laat toe om links te genereren tussen verschillende xml bestanden.
Links worden aangegeven door de link tag, waarna in de 'value' van het element een link kan worden toegevoegd tussen '� �'.  
Zie voorbeeld hieronder:

\begin{figure}[H]%
\includegraphics[width=\columnwidth]{XML2.jpg}%
\caption{Link tussen xml files}%
\end{figure}

Na het inladen van File1.xml heeft 'element' de waarde 'The selected language is Dutch'.

%\paragraph{Elementen}
%
%\begin{itemize}
%\item[-] \textbf{XMLDocument}: Object verwijzende naar een XML bestand. Een XMLDocument bevat een lijst van 	XMLElement \\[-5mm]
%\item[-] \textbf{XMLElement}: Element van een XML bestand. Een element heeft een bepaalde waarde, of bevat zelf een 	aantal andere elementen. \\[-5mm]
%\begin{itemize}
%	\item[-] \textbf{ElementWithValue}: Element met een waarde \\[-5mm]
%	\item[-] \textbf{ElementWithChildren}: Element met lijst van andere elementen, zijn \textit{Children}.(\textit{Children} kunnen op hun beurt ook weer \textit{ElementWithChildren} of \textit{ElementWithValue} zijn) \\[-5mm]
%\end{itemize}		
%\item[-] \textbf{XMLParser}: Levert methodes om een XML bestand, gecre�erd met deze interface, weer in te laden naar 	een XMLDocument 
%\end{itemize}
%
%\paragraph{Schema}
%
%\begin{figure}[H]
%\includegraphics[width=\columnwidth]{XML.png}%
%\caption{Voorbeeld van een \textit{XMLDocument}}%
%\end{figure}

\newpage

%-----------------------------------------------------------------------------------------------------------------------------------------------------------------

\subsection{Design viewpoint 4: Algoritme voor kalenderplanning}
% Matthias schrijft dit wel
\subsubsection{Design concerns}

Het algoritme dient in staat te zijn om een lessenrooster te maken dat aan bepaalde voorwaarden (constraints) voldoet. Het is daarom belangrijk dat deze eerst bepaald worden. Men kan een indeling maken in deze voorwaarden. Zo zijn er de \textit{fixed} constraints aan deze moeten zeker voldaan zijn, anders klopt het lessen rooster niet. \textit{Hard constraints} deze moeten zo goed mogelijk vervuld zijn en dan zijn er nog \textit{soft constraints} deze staan onderaan de ladder en hoeven dus niet noodzakelijk vervuld te zijn. Aan de hand van deze voorwaarden kan men dan nagaan hoe goed het lessenrooster is. Dit doet men aan de hand van een \textit{fitnessvalue}. Zo krijgt een geplaatste les 3 punten als aan een \textit{fixed constraint} voldaan is, 2 voor een hard en 1 voor een soft. Aangezien de \textit{fixed constraints} zeker voldaan moeten zijn, moet een geplaatste les al zeker 12 punten hebben. Op die manier kan men het resultaat evalueren.

\begin{itemize}
\item[] \textit{Fixed constraints}\\[-3mm]
\begin{enumerate}
	\item Er mag nooit meer dan 1 les gepland zijn in een leslokaal \\[-3mm]
	\item Een docent (educator) kan niet meer dan 1 les geven  \\[-3mm]
	\item Een leslokaal (room) moet groot genoeg zijn voor het aantal studenten \\[-3mm]
	\item Als een les bepaald materiaal vereist (zoals computer, laboratorium,...) moet hier aan worden tegemoet gekomen. \\[-3mm]
\end{enumerate}
\item[] \textit{Hard constraints} \\[-3mm]
\begin{enumerate}
 \item Het lessenrooster van een student mag niet overlappen. \\[-3mm]
 \item Er mag van een vak bijvoorbeeld hoogstens 4 uur per dag gegeven worden. \\[-3mm]
 \item Student mogen bijvoorbeeld maximum 8 uur les per dag krijgen. \\[-3mm]
\end{enumerate}
\item[] \textit{Soft constraints} \\[-3mm]
\begin{enumerate}
 \item De werkcolleges mogen pas na of gelijktijdig met de theoriecolleges beginnen. \\[-3mm]
\end{enumerate}
\end{itemize}

Voor iedere week van het academisch jaar zal de lessenrooster een tabel bevatten voor elke student die bestaat uit 5 dagen (maandag t.e.m. vrijdag) waar voor elke dag de uren bijstaan (er kan les gegeven worden van 8u t.e.m. 18u). \\

Het algoritme zal deze tabel in een eerste fase zo opbouwen dat er aan de fixed constraints voldaan is en dan in een tweede fase proberen om de totale fitnessvalue op te drijven.

\subsubsection{Design elements}

Om het algoritme te kunnen uitvoeren is er natuurlijk de nodige informatie nodig om tot een lessenrooster te kunnen komen. De informatie die nodig is, zal uit een database worden gehaald en dan in klassenstructuur gegoten worden. Deze klassenstructuur is besproken in "Design Viewpoint 2: Logical".
Het schema en bijbehorende uitleg kan daar gevonden worden.
Deze structuur zal nog worden uitgebreid met methode die de rooster opstellen en dan hieruit vertrekkende de fitnessvalue bepalen. De bovenstaande structuur zal met andere woorden nog worden uitgebreid.

%\subsection{Design rationale}


%\begin{figure}
%\begin{center}
%\includegraphics[width=\textwidth]{Bangladesh}
%\caption*{Bangladesh}
%\end{center}
%\end{figure}

%\begin{tabular}[t]{llll}
%Mandi & Katholiek & 25\% & Platte neus -- spleetogen \\
%Gohli & Moslims   & 75\% & Grote ogen -- scherpere neus \\
%Kooch & Hindu     & 1\%  & Mengeling van de twee \\
%\end{tabular}
%\\[5mm]

%\begin{itemize}
%\item[.] Slaapmatje \\[-3mm]
%\item[.] Kleren: 3 korte broeken, 6 onderbroeken, 4 T-shirts\\[-3mm]
%\item[.] Longi: voor 's avonds en als pyjama\\[-3mm]
%\item[.] Tandenborstel en twee tubes tandpasta\\[-3mm]
%\end{itemize}


 \end{document}