\documentclass{article}
\usepackage[dutch]{babel}
\usepackage[official]{eurosym}
%\usepackage{graphicx}
%\usepackage{epic,eepic}
\usepackage{mathtools}
\usepackage{caption}
\usepackage{textcomp}
\title{Software Requirements Specifications voor Schedule-Generator}
\author{Matthias Caenepeel \and Adam Cooman \and Alexander De Cock \and Zjef Van de Poel}
\date{23 februari 2011 Versie 0.1} 

\begin{document}

\maketitle

\newpage

% \newpage
% Signature page

% \newpage

\section*{Aanpassingsgeschiedenis}
\begin{itemize}
\item[.] 23/2/2011 versie 0.1: Aanmaak document, toevoeging  \\[-3mm]
\item[.] 25/2/2011 versie 0.2: Toevoeging tekst vanaf Performance Requirements tot einde
\end{itemize}

%\section*{Preface}

\newpage
\tableofcontents

\newpage

\section{Introduction}
\subsection{Purpose}
% Delineate the purpose of the SRS.
% Specify the intended audience for the SRS.

Het doel van de SRS is om een overzicht te geven van alle functionaliteiten die er moeten voorzien worden. \\
Het doelpubliek is het team dat aan het project werkt en de docent die het team begeleidt en evalueert. \\

\subsection{Scope}
% Identify the software product(s) to be produced by name (e.g., Host DBMS, Report Generator, etc.);
% Explain what the software product(s) will, and, if necessary, will not do;
% Describe the application of the software being specified, including relevant benefits, objectives, and goals;
% Be consistent with similar statements in higher-level specifications (e.g., the system requirements speci�cation), if they exist.

Schedule generator: deze software (geschreven in Java) zal in staat zijn om een lessenrooster samen te stellen dat aan bepaalde voorwaarden voldoet.  \\
Website: dit is de software die de site omvat waarop de gebruikers zullen werken. \\
Database: deze zal de informatie over het lessenrooster bijhouden en ordenen. \\
Servelets: deze software zal ervoor zorgen dat de site doet wat de gebruiker vraagt. 

%\subsection{Definitions, acronyms and abbreviations}
\subsection{references}
%Provide a complete list of all documents referenced elsewhere in the SRS;
%Identify each document by title, report number (if applicable), date, and publishing organization;
%Specify the sources from which the references can be obtained.

IEEE Recommended Practice for Software Requirements Specifications
De organisation pdf die de opdracht bevat.

\subsection{Overview}
% Describe what the rest of the SRS contains;
% Explain how the SRS is organized.

De rest van de SRS zal de software vereisten verder uitdiepen.

\newpage

\section{Overall description}
% \subsection{Product perspective}

% blokdiagram van programma en verschillende softwareinterfaces die we gebruiken
% MySQL, Tomcat, Java Runtime Enviroment voor generator zelf ...

% \subsubsection{System interfaces}
% \subsubsection{User interfaces}
% \subsubsection{Hardware interfaces}
% \subsubsection{Software interfaces}
% \subsubsection{Communications interfaces}
% \subsubsection{Memory}
% \subsubsection{Operations}
% \subsubsection{Site adaptation requirements}

\subsection{Product functions}
%samenvatting van de belangrijkste functionaliteiten. \\

Gebruikers krijgen een gepersonaliseerde account waarop zij hun eigen lessenrooster alsook dat van andere richtingen kunnen opvragen (en in het geval van sommige gebruikers kunnen aanpassen).\\
Het programma bevat een database waarin informatie over de verschillende vakken wordt bijgehouden (wie is de docent, hoeveel studenten zijn er ingeschreven, waar wordt het gedoceerd,...)\\
Het bevat ook een schedule generator die in staat is een lessenrooster te genereren die aan bepaalde voorwaarden voldoet.

\subsection{User characteristics}

De gebruikers zijn studenten die in staat om zijn om de verschillende lessenroosters te bekijken en docenten die in staat zijn het rooster op te vragen, maar evenwel aanpassingen te maken aan de informatie van de vakken waarvoor zij verantwoordelijk zijn.

\subsection{constraints}

Het geheel moeten worden gedraaid op de server Wilma die als besturingssysteem Linux heeft.\\
Alle code moet open source zijn.\\
Er moet geprogrammeerd worden in Java.\\
Men heeft internet nodig om aan de diensten te kunnen.

% \subsection{Assumptions and dependencies}

\newpage

\section{Specific requirements}

\subsection{External interface requirements}

\subsubsection{User interfaces}

De opdracht gebiedt ons om met een website als user interface te werken. Om dit te verwezenlijken zal XHTML in combinatie met CSS gebruikt worden (mogelijk ook javascript). 
Na het inloggen krijgt de gebruiker een pagina te zien met verschillende tabs. Voor elke gebruikersklasse en de functionaliteiten die ze krijgen bestaat een ander tabblad. Op die manier is het gemakkelijker om functionaliteiten toe te voegen en overzichtelijk weer te geven. Guests krijgen bijvoorbeeld maar een enkele tab te zien waarin ze een naam kunnen intypen en waarin de kalender weergegeven wordt. Studenten krijgen een tweede tab erbij waarin ze hun vakkenlijst kunnen aanpassen etc.

\subsubsection{Communications and software interfaces}

Om de communicatie tussen de browser van de gebruiker en de server te doen wordt gebruik gemaakt van HTTP. De HTTP paketten zullen op de server ge\"{i}nterpreteerd worden door servlets die op de server uitgevoerd worden via Tomcat. De servlets zijn in JAVA geschreven en genereren de gevraagde XHTML en CSS code die dat terug naar de gebruiker gestuurd worden.\\ 

De servlets communiceren met de database via MySQL en interpreteren de gegevens voor de gebruiker. Het proces kan overzichtelijk weergegeven worden in volgend blokschema:

\unitlength=0.3\textwidth
\begin{center}
\begin{picture}(3,1)
%serverlijnen
\put(0.2,0.2){Server}
\put(0,0){\line(1,0){1.6}}
\put(0,0){\line(0,1){1}}
\put(1.6,0){\line(0,1){1}}
\put(0,1){\line(1,0){1.6}}
%browser
\put(2.4,0.6){Browser}
\put(2.3,0.9){\line(1,0){0.5}}
\put(2.8,0.4){\line(0,1){0.5}}
\put(2.3,0.4){\line(0,1){0.5}}
\put(2.3,0.4){\line(1,0){0.5}}
%database
\put(0.15,0.6){Database}
\put(0.1,0.9){\line(1,0){0.5}}
\put(0.6,0.4){\line(0,1){0.5}}
\put(0.1,0.4){\line(0,1){0.5}}
\put(0.1,0.4){\line(1,0){0.5}}
%servlet
\put(1.1,0.6){Servlet}
\put(1.0,0.9){\line(1,0){0.5}}
\put(1.5,0.4){\line(0,1){0.5}}
\put(1.0,0.4){\line(0,1){0.5}}
\put(1.0,0.4){\line(1,0){0.5}}
%lijn MySQL
\put(0.65,0.7){MySQL}
\put(0.62,0.65){\line(1,0){0.36}}
%lijn HTTP
\put(1.8,0.7){HTTP}
\put(1.52,0.65){\line(1,0){0.76}}
\end{picture}\\[3mm]
\caption{Overzicht van de communicatie en de gebruikte protocols tussen de verschillende delen van het programma}
\end{center}

De verschillende versies software die we gebruiken zijn de volgende:

\begin{itemize}
\item[.] MySQL: JDBC driver for MySQL 5.1.15 \\[-5mm]
\item[.] Tomcat \\[-5mm]
\end{itemize}

\subsubsection{Hardware interfaces}

De Hardware interfaces die we gebruiken worden vastgelegd door de opdrachtgever. Ons programma moet op Wilma kunnen draaien. De hardware interface die we hebben is dus een Linux besturingssysteem.

\newpage
\subsection{Functional requirements}

ALEXANDER

\subsubsection{User class 0: Unknown}

login  \\
enter as guest  \\
Link naar email van beheerder

\subsubsection{User class 1: Guest}

Bekijk rooster van student, educator, lokaal \\
Bekijk eigenschappen van Vakken \\
Bekijk eigenschappen van Studenten \\
Bekijk eigenschappen van Proffen \\
Bekijk eigenschappen van Lokalen \\
Link naar email van beheerder

\subsubsection{User class 2: Student}

Breidt de functionaliteit van guest uit (kan kijken naar rooster) \\
Vakkenlijst aanpassen (door program toe te voegen of enkele vakken) \\
kan gegevens aanpasse (login naam, paswoord, ...) \\

\subsubsection{User class 3: Educator}

Breidt ook guest uit (kan kijken naar rooster)

Vakken aanpassen (enkel die die hij zelf geeft), constraints, description,...
Beschikbaarheid aanpassen

\subsubsection{User class 4: Facility Manager}

Kan Lokalen aanpassen \\
Kan Lokalen toevoegen/verwijderen \\
...

\subsubsection{User class 5: Program Manager}

Kan proffen op vakken zetten \\
Kan vakken per programma aanpassen \\
Kan vakken toevoegen \\

\subsubsection{User class 6: Account Manager}

Kan accounts wijzigen/toevoegen/verwijderen,... van Student en Educator

\subsubsection{User class 7: Secretaresse}

Heeft elke functionaliteit van guest, Program Manager, Account Manager, Facility Manager \\
Kan Rooster aanpassen

\subsubsection{User class X: Admin}

Kan alles \\
genereren \\
Kan Managers maken


\subsubsection{Opmerking over User Classes}

We verkiezen een methode waarbij de rechten van een account niet volledig vast liggen door zijn user class, maar veel losser kunnen bepaald worden. De user classes uit de vorige puntjes zijn templates, richtwaarden waar van afgeweken kan worden. De structuur moet dus per feature beschreven worden in een latere versie van het SRS.

\\ 
\\

%ZJEF

\subsection{Performance requirements}

Er zijn geen specifieke vereisten in verband met de snelheid van de software. Het is echter wel duidelijk uit de aard van het project, dat er een groot aantal gebruikers tegelijk de webinterface (en met gevolg de databases) moet kunnen consulteren.

\subsection{Design constraints}

De opdrachtgever heet de omgeving waarin de software moet draaien gespecificeerd.\\
De opgelegde beperkingen, met betrekking op het design, zijn de volgende:

\begin{itemize}
\item[.] Systeem moet draaien op een linux server, meer bepaald Wilma (http://wilma.vub.ac.be/) \\[-5mm]
\item[.] Uitsluitend gebruik van open source software \\[-5mm]
\item[.] User interactie gebeurt via een gebruiksvriendelijke grafische web interface met bovengenoemde server \\[-5mm]
\item[.] Flexibiliteit van verschillende parameters instelbaar door de gebruiker \\[-5mm]
\end{itemize}

%\subsection{Software system attributes}

% \subsubsection{Reliability}
% This should specify the factors required to establish the required reliability of the software system at time of delivery.


% \subsubsection{Availability}
% This should specify the factors required to guarantee a defined availability level for the entire system such as checkpoint, recovery, and restart.


\subsubsection{Security}
%This should specify the factors that protect the software from accidental or malicious access, use, modification, destruction, or disclosure. Specific requirements in this area could include the need to
% a) Utilize certain cryptographical techniques;
% b) Keep specific log or history data sets;
% c) Assign certain functions to different modules;
% d) Restrict communications between some areas of the program;
% e) Check data integrity for critical variables.

%Als de gebruiker zich aanmeldt, wordt een code gegenereerd voor die gebruiker (random nummer) die lokaal bijgehouden wordt door de gebruiker.\\
%Bij elk commando van de gebruiker wordt de code meegegeven.\\
%Op de server wordt een tijdelijke lijst bijgehouden die de code aan accounts verbindt. (en inlogtijd,... om ze na een tijd weg te kunnen smijten als de gebruiker niet letterlijk uitlogt). \\
%Bij elk commando van de gebruiker wordt de code in de lijst opgezocht, de rechten van de bijhorende gebruiker gecontroleerd en al dan niet kan het commando uitgevoerd worden.\\[5mm]
%Logboek van aanpassingen door Managers wordt bijgehouden.\\

\textbf{User access restriction}

Verschillende gebruikers krijgen verschillende rechten toegewezen. Deze bepalen tot welke informatie en tools hij toegang krijgt. Deze rechten zijn gekoppeld aan de account van deze gebruiker.
Na het inloggen zullen enkel de informatie en tools waarvoor de gebruiker gemachtigd is, getoond worden. 
Om verdere beveiliging te verzekeren, wordt ook de communicatie met de server beveiligd. 
Bij het inloggen krijgt de webinterface van de gebruiker een access code toegestuurd, op dat moment gegenereerd door de server. Deze houdt bij welke rechten bij deze code horen. De webinterface stuurt de verkregen code mee door met elke instructie naar de server, waarop deze kan controleren of de gebruiker gemachtigd is om de desbetreffende instructie uit te voeren.\\

\textbf{Data integriteit}

Enerzijds zal er de mogelijkheid geleverd worden, aan de daarvoor gemachtigde gebruiker(s), om op de server een back-up van de databases (zoals accounts, leslokalen, vakken,...) te maken en desnoods een rollback uit te voeren.
Anderzijds zal er op de server een logboek bijgehouden worden met de aanpassingen aan de databases, die door verschillende gebruikers gemaakt kunnen worden.\\

\textbf{Account security}

Account paswoorden zullen, met een nog nader te bepalen, ge\"{e}ncrypteerd verstuurd en opgeslagen
worden.

% \subsubsection{Maintainability}
% This should specify attributes of software that relate to the ease of maintenance of the software itself. There may be some requirement for certain modularity, interfaces, complexity, etc. Requirements should not be placed here just because they are thought to be good design practices.


% \subsubsection{Portability}
% This should specify attributes of software that relate to the ease of porting the software to other host machines and/or operating systems. This may include the following:
% a) Percentage of components with host-dependent code;
% b) Percentage of code that is host dependent;
% c) Use of a proven portable language;
% d) Use of a particular compiler or language subset;
% e) Use of a particular operating system.

% \subsection{Other requirements}

%\newpage
%\section*{Index}




%\begin{figure}
%\begin{center}
%\includegraphics[width=\textwidth]{Bangladesh}
%\caption*{Bangladesh}
%\end{center}
%\end{figure}

%\begin{tabular}[t]{llll}
%Mandi & Katholiek & 25\% & Platte neus -- spleetogen \\
%Gohli & Moslims   & 75\% & Grote ogen -- scherpere neus \\
%Kooch & Hindu     & 1\%  & Mengeling van de twee \\
%\end{tabular}
%\\[5mm]

%\begin{itemize}
%\item[.] Slaapmatje \\[-3mm]
%\item[.] Kleren: 3 korte broeken, 6 onderbroeken, 4 T-shirts\\[-3mm]
%\item[.] Longi: voor 's avonds en als pyjama\\[-3mm]
%\item[.] Tandenborstel en twee tubes tandpasta\\[-3mm]
%\end{itemize}


 \end{document}
